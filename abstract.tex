\begin{abstract}
\begin{center}
\LARGE Enhanced Fidelity of Monte Carlo Coupled Multi-Physics Simulations for Light Water Reactors 
\end{center}
\vspace{1cm}

This thesis presents a framework to enhance the fidelity of typical Monte Carlo (MC) coupled multi-physics simulations for Light Water Reactors (LWRs) through two key improvements. First, the introduction of multi-physics simulations with spatially continuous material properties using the Functional Expansion Tally (FET) combined with delta-tracking. Second, the incorporation of on-the-fly thermal expansion of reactor core components during MC particle tracking. The use of spatially continuous material properties is  particularly crucial to accurately modeling spatial self-shielding effects, which depend on smooth intra-fuel-pellet temperature distributions. Additionally, modeling thermal expansion is essential because the geometric data of reactor components is typically provided at room temperature, while reactors operate at much higher temperatures. Numerical experiments are conducted to evaluate the proposed multi-physics framework across a range of reactor core problems, from two-dimensional pin-cell to whole-core reactor problems. Numerical results showed that introducing spatially continuous material properties alone produces solutions that asymptotically approach those from conventional cell-based discretized simulations with infinitesimally small cells. Additionally, the method reduces simulation times by over threefold compared to the cell-based discretized approach with very small cells for typical LWR problems. Moreover, incorporating both spatially continuous material properties and thermal expansion significantly improves the accuracy of reactor simulations when compared to measured reactor data. These findings suggest that integrating the proposed framework into reactor modeling can significantly enhance simulation fidelity. Therefore, the framework has the potential to be incorporated into future Monte Carlo production codes to meet the growing demands for improved reactor safety.
\vfill
\end{abstract}

\clearpage