\begin{abstract}
% \begin{center}
% \large Enhanced Fidelity of Monte Carlo Coupled Multi-Physics Simulations for Light Water Reactors 
% \end{center}
\vspace{1cm}

This study presents a framework to enhance the fidelity of typical Monte Carlo (MC) coupled multi-physics simulations for Light Water Reactors (LWRs) through two key improvements. First, the introduction of multi-physics simulations with spatially continuous material properties using the Functional Expansion Tally (FET) combined with delta-tracking. Second, the incorporation of on-the-fly thermal expansion of reactor core components during MC particle tracking. In direct multi-physics coupled MC simulations, the use of spatially continuous material properties is  particularly crucial to accurately modeling spatial self-shielding effects, which depend on smooth intra-fuel-pellet temperature distributions. The significant reduction in discretization also preserves the MC method's advantage in handling continuous geometry. Additionally, modeling thermal expansion is essential because the geometric data of reactor components is typically provided at room temperature, while reactors operate at much higher temperatures. Numerical experiments are conducted to assess the applicability and advantages of the proposed multi-physics framework across a range of reactor core problems, from two-dimensional pin-cell to whole-core reactor problems.

The incorporation of spatially continuous material properties produces solutions that asymptotically converge to those from conventional cell-based discretized simulations with infinitesimally small cells as demonstrated in the two-dimensional pin-cell problem. Similar outcomes are observed in three-dimensional pin cell and assembly problems, where the continuous representation of material properties results in more accurate solutions for both eigenvalue and axial power distributions, compared to the conventional cell-based discretization method. In typical whole-core LWR problems, the proposed method reproduces high-fidelity solutions for both eigenvalue and pin powers, while reducing simulation times by around threefold and requiring 80\% less memory than the traditional cell-based discretization using very small cells.

Whereas the numerical results for on-the-fly thermal expansion demonstrate that the observed trends in reactivity differences due to thermal expansion for varying boron concentrations and fuel temperatures align with previous studies. Additionally, the calculated Isothermal Temperature Coefficients (ITC) show improvement when thermal expansion is considered. Using core-averaged temperatures for expansion provides fairly accurate results, but employing local temperatures at pin levels can enhance accuracy further. Moreover, incorporating thermal expansion also improves solutions for depletion problems, especially at high power and high fuel burnup. These findings suggest that integrating the proposed framework into reactor modeling can significantly enhance simulation fidelity. Therefore, the framework has the potential to be incorporated into future Monte Carlo production codes to meet the growing demands for improved reactor safety.

\vfill
\end{abstract}

\clearpage