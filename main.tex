%%%%%%%%%%%%%%%%%%%%%%%%%%%%%%%%%%%%%%%%%%%%%%%%%%%%%%%%%%%%%%%%%%
%%%%%%%%%%    Do not touch the below 70 lines     %%%%%%%%%%%%%%%%
%%%%%%%%%%%%%%%%%%%%%%%%%%%%%%%%%%%%%%%%%%%%%%%%%%%%%%%%%%%%%%%%%%
\documentclass[11pt,a4paper,onecolumn,oneside]{report}

\usepackage{mathptmx}
\usepackage[T1]{fontenc}
\usepackage[utf8]{inputenc}

\RequirePackage[top=3cm, bottom=1in, left=1in, right=1in]{geometry}
\linespread{1.3}

\usepackage{titlesec}
\usepackage{amsmath}
\usepackage{amssymb}
\usepackage{mathtools}
\usepackage{enumerate}
\usepackage{bbm}
\usepackage{algorithm}
\usepackage{algorithmic}
\usepackage{epsfig}
\usepackage{color}
\usepackage{graphicx}
\usepackage{caption}
\usepackage{subcaption}
\usepackage{cases}
\usepackage{url}
\usepackage{cite}
\usepackage{fancyhdr}
\usepackage{tocloft}
\usepackage{pdfpages}
\usepackage{adjustbox}
\usepackage{indentfirst}
\usepackage{tocloft}

\newlength{\mylen}

\renewcommand{\cftfigpresnum}{\figurename\enspace}
\renewcommand{\cfttabpresnum}{\tablename\enspace}
\renewcommand{\cftfigaftersnum}{:}
\renewcommand{\cfttabaftersnum}{:}
\settowidth{\mylen}{\cftfigpresnum\cftfigaftersnum}
\settowidth{\mylen}{\cfttabpresnum\cfttabaftersnum}
\addtolength{\cftfignumwidth}{\mylen}
\addtolength{\cfttabnumwidth}{\mylen}

\setcounter{secnumdepth}{3}
\setcounter{tocdepth}{3}

\renewcommand\cftsecafterpnum{\vskip15pt}
\renewcommand\cftsubsecafterpnum{\vskip15pt}
\renewcommand\cftfigafterpnum{\vskip15pt}
\renewcommand{\thesection}{\Roman{section}.}
\renewcommand{\thesubsection}{\arabic{section}.\arabic{subsection}.}
\renewcommand{\thesubsubsection}{\arabic{section}.\arabic{subsection}.\arabic{subsubsection}.}
\renewcommand{\contentsname}{\hfill\bfseries\Large Contents\hfill}   
\renewcommand{\listfigurename}{\hfill\bfseries\Large List of Figures\hfill}
\renewcommand{\listtablename}{\hfill\bfseries\Large List of Tables\hfill}
\renewcommand{\thefigure}{\arabic{figure}}
\newcommand{\qed}{\hfill\blacksquare}
\renewcommand{\bibname}{\hfill\bfseries\Large References \hfill\hfill}
\renewcommand{\abstractname}{\bfseries\Large Abstract \hfill\hfill}

\captionsetup{figurename=Figure,labelsep=period}
\captionsetup{tablename=Table,labelsep=period}

\hyphenpenalty=500

\newcounter{lemma}
\newcounter{proposition}
\newcounter{theorem}
\newtheorem{lemma}{\bf Lemma}
\newtheorem{proposition}{\bf Proposition}
\newtheorem{theorem}{\bf Theorem}
\newtheorem{proof}{\bf Proof}

%\input{mymath_mod.tex}


\newcommand{\HIGH}[1]{{\textcolor{blue}{#1}}}
%\renewcommand{\baselinestretch}{1.5} 

\DeclareMathOperator*{\argmax}{arg\,max}

\fancyhf{}
\renewcommand{\headrulewidth}{0pt}
\cfoot{\thepage}
\pagestyle{fancy}
%\pagenumbering{gobble}

\begin{document}

%%%%%%%%%%%%%%%%%%%%%%%%%%%%%%%%%%%%%%%%%%%%%%%%%%%%%%%%%%%%%%%%%%
%%%%%%%%%%    Do not touch the above 70 lines     %%%%%%%%%%%%%%%%
%%%%%%%%%%%%%%%%%%%%%%%%%%%%%%%%%%%%%%%%%%%%%%%%%%%%%%%%%%%%%%%%%%


%%%%%%%%%%%%%%%%%%%%%%%%
%%%%%% Front cover
%%%%%%%%%%%%%%%%%%%%%%%%
\begin{center}
\LARGE Doctoral Thesis
% or Ph.D...

\vspace{3cm}
\huge Enhanced Fidelity of Monte Carlo Coupled Multi-Physics Simulations for Light Water Reactors

\vfill

\LARGE Muhammad Imron

\vspace{2cm}

\LARGE Department of Nuclear Engineering 
% Department of Computer Science and Engineering

\vspace{2cm}

\LARGE Ulsan National Institute of Science and Technology
\vspace{2cm}

\LARGE 2024

\end{center}
\thispagestyle{empty}
\clearpage

%%%%%%%%%%%%%%%%%%%%%%%%
%%%%%% TITLE PAGE
%%%%%%%%%%%%%%%%%%%%%%%%
\begin{center}
\hbox{ }

\hbox{ }

\huge Enhanced Fidelity of Monte Carlo Coupled Multi-Physics Simulations for Light Water Reactors 

\vspace{5cm}

\LARGE Muhammad Imron

\vspace{6cm}

\LARGE Department of Nuclear Engineering 

\vspace{2cm}

\LARGE Ulsan National Institute of Science and Technology

\end{center}
\thispagestyle{empty}
\clearpage

%%%%%%%%%%%%%%%%%%%%%%%%%%%%%%%%%%%%%%%%%%%%%
%			THESIS APPROVAL
%%%%%%%%%%%%%%%%%%%%%%%%%%%%%%%%%%%%%%%%%%%%%

% [Thesis approval]
% Add the approval doc signed by your advisor in a PDF file
% Put your pdf with the filename below, and uncomment it.
%\includepdf[fitpaper= true, pages=-]{sample_approval.pdf}

% [Confirmation of thesis approval]
% add the certificate signed by your committee in a PDF file
% Put your pdf with the filename below, and uncomment it.
%\includepdf[fitpaper= true, pages=-]{sample_confirmation.pdf}

%%%%%%%%%%%%%%%%%%%%%%%%%%%%%%%%%%%%%%%%%%%%%
%			ABSTRACT
%%%%%%%%%%%%%%%%%%%%%%%%%%%%%%%%%%%%%%%%%%%%%
\begin{abstract}
% \begin{center}
% \large Enhanced Fidelity of Monte Carlo Coupled Multi-Physics Simulations for Light Water Reactors 
% \end{center}
\vspace{1cm}

This study presents a framework to enhance the fidelity of typical Monte Carlo (MC) coupled multi-physics simulations for Light Water Reactors (LWRs) through two key improvements. First, the introduction of multi-physics simulations with spatially continuous material properties using the Functional Expansion Tally (FET) combined with delta-tracking. Second, the incorporation of on-the-fly thermal expansion of reactor core components during MC particle tracking. In direct multi-physics coupled MC simulations, the use of spatially continuous material properties is  particularly crucial to accurately modeling spatial self-shielding effects, which depend on smooth intra-fuel-pellet temperature distributions. The significant reduction in discretization also preserves the MC method's advantage in handling continuous geometry. Additionally, modeling thermal expansion is essential because the geometric data of reactor components is typically provided at room temperature, while reactors operate at much higher temperatures. Numerical experiments are conducted to assess the applicability and advantages of the proposed multi-physics framework across a range of reactor core problems, from two-dimensional pin-cell to whole-core reactor problems.

The incorporation of spatially continuous material properties produces solutions that asymptotically converge to those from conventional cell-based discretized simulations with infinitesimally small cells as demonstrated in the two-dimensional pin-cell problem. Similar outcomes are observed in three-dimensional pin cell and assembly problems, where the continuous representation of material properties results in more accurate solutions for both eigenvalue and axial power distributions, compared to the conventional cell-based discretization method. In typical whole-core LWR problems, the proposed method reproduces high-fidelity solutions for both eigenvalue and pin powers, while reducing simulation times by around threefold and requiring 80\% less memory than the traditional cell-based discretization using very small cells.

Whereas the numerical results for on-the-fly thermal expansion demonstrates that the observed trends in reactivity differences due to thermal expansion for varying boron concentrations and fuel temperatures align with previous studies. Additionally, the calculated Isothermal Temperature Coefficients (ITC) show improvement when thermal expansion is considered. Using core-averaged temperatures for expansion provides fairly accurate results, but employing local temperatures at pin levels can enhance accuracy further. Moreover, incorporating thermal expansion also improves solutions for depletion problems, especially at high power and high fuel burnup. These findings suggest that integrating the proposed framework into reactor modeling can significantly enhance simulation fidelity. Therefore, the framework has the potential to be incorporated into future Monte Carlo production codes to meet the growing demands for improved reactor safety.

\vfill
\end{abstract}

\clearpage

%%%%%%%%%%%%%%%%%%%%%%%%%%%%%%%%%%%%%%%%%%%%%%%%%%%%%%%%%%%%% 
%%%%%%%%%%    Do not touch the below 20 lines 
%%%%%%%%%%%%%%%%%%%%%%%%%%%%%%%%%%%%%%%%%%%%%%%%%%%%%%%%%%%%% 
%%% The following page is intentionally left as blank
%%% White attachment form
\hbox{ }
\thispagestyle{empty}
\clearpage

%%% Table of Contents
\tableofcontents{}
\thispagestyle{empty}
\vfill
\clearpage

%%% List of Figures
\listoffigures
\thispagestyle{empty}
\clearpage

%%% List of Tables
\listoftables
\thispagestyle{empty}
\clearpage

%%% reset page numbering
\setcounter{page}{1}
%%%%%%%%%%%%%%%%%%%%%%%%%%%%%%%%%%%%%%%%%%%%%%%%%%%%%%%%%%%%% 
%%%%%%%%%%    Do not touch the above 20 lines 
%%%%%%%%%%%%%%%%%%%%%%%%%%%%%%%%%%%%%%%%%%%%%%%%%%%%%%%%%%%%% 



%%%%%%%%%%%%%%%%%%%%%%%%%%%%%%%%%%%%%%%%%%%%%
%			Introduction
%%%%%%%%%%%%%%%%%%%%%%%%%%%%%%%%%%%%%%%%%%%%%
\section{Introduction}
Advancements in modern computing over the past 50 years have enabled the direct whole-core Monte Carlo coupled multi-physics simulations of large-scale light water reactors. Numerous examples of such simulations can be found in the literature. However, even with such "high-fidelity" methodology, their solutions still often cannot reflect the experimental solutions. The reasons could be many, but two of them are: first, typical direct whole-core Monte Carlo coupled multi-physics simulations often ignore intra-fuel-pellet temperature distribution which particularly crucial for correctly modeling spatial self-shielding effects. Second, typical direct whole-core Monte Carlo coupled multi-physics simulations often ignore thermal expansion of the reactor core materials.

Your introduction should be here. You may need a reference~\cite{ref_sample}.


%%%%%%%%%%%%%%%%%%%%%%%%%%%%%%%%%%%%%%%%%%%%%
%			section 2
%%%%%%%%%%%%%%%%%%%%%%%%%%%%%%%%%%%%%%%%%%%%%
\newpage 
\section{Spatially Continuous Material Properties} 

My second section starts with my equation, which can be written as 
%
\begin{equation}\label{eq:myeq}
\begin{split}
	A 		&= B + C, \\
    D + E	&= F.
\end{split}
\end{equation}
Thus we have (\ref{eq:myeq}).

Fig.~\ref{fig:myfigure} is a sample figure. 

\begin{figure}[h]
\centering
\includegraphics[width=5in]{myfigure.pdf}
\caption{My figure.} \label{fig:myfigure}
\end{figure}

%%%% use the following in the case of eps figure
%\begin{figure} 
%    \centering
%    \epsfig{file=model.eps, width = 0.9\linewidth}
%    \caption{System model with complete bipartite graph. The maximum weighted matching is marked by circles.}
%    \label{fig:model}
%\end{figure}

\subsection{Functional Expansion Tally}
\subsection{Polynomials Calculation}
\subsection{Delta-tracking}
\subsection{Majorant Cross Section Computation}
\subsection{Calculation Flow}


%%%%%%%%%%%%%%%%%%%%%%%%%%%%%%%%%%%%%%%%%%%%%
%			section 3
%%%%%%%%%%%%%%%%%%%%%%%%%%%%%%%%%%%%%%%%%%%%%
\newpage 
\section{Thermal Expansion} 
My third section starts. 

\begin{algorithm}
	\caption{My Algorithm.} \label{algo:myalgo}
    At the beginning ...
    
	\begin{algorithmic}[1]	    
	    \STATE Do this
        \STATE and do this\\
        /* add explanation if necessary */
        \STATE Finally do this
	\end{algorithmic} 
\end{algorithm}
%
Algorithm~\ref{algo:myalgo} is our proposed algorithm.

\begin{table}[!b]
  \centering
  \caption{Solutions for the criticality calculations}
  \label{tab1} 
\begin{adjustbox}{width=1.0\textwidth} % Adjust your table to the text width
  \begin{tabular}{| p{0.05\linewidth} | p{0.1\linewidth} | p{0.1\linewidth} | p{0.1\linewidth} | c | c | p{0.1\linewidth} |}
  \hline 
         &             &                 &                     & \multicolumn{2}{c|}{$k_{eff}$}             &            \\
   \cline{5-6}
   Cases & Boron (ppm) & Bank D Position (steps) & Fully inserted bank & MCS                   & KENO                  & Difference (pcm) \\
   \hline
   1     & 1285        & 167             & -                   & 0.99954 $\pm$ 0.00002 & 0.99990 $\pm$ 0.00001 & -36 $\pm$ 2   \\ \hline
   2     & 1291        & 230             & -                   & 0.99998 $\pm$ 0.00003 & 1.00032 $\pm$ 0.00001 & -34 $\pm$ 3   \\ \hline
   3     & 1270        & 97              & Bank A              & 0.99836 $\pm$ 0.00003 & 0.99880 $\pm$ 0.00001 & -44 $\pm$ 3   \\ \hline
   4     & 1270        & 113             & Bank B              & 0.99900 $\pm$ 0.00003 & 0.99936 $\pm$ 0.00001 & -36 $\pm$ 3   \\ \hline
   5     & 1270        & 119             & Bank C              & 0.99863 $\pm$ 0.00003 & 0.99904 $\pm$ 0.00001 & -41 $\pm$ 3   \\ \hline
   6     & 1270        & 18              & Bank D              & 0.99871 $\pm$ 0.00003 & 0.99908 $\pm$ 0.00001 & -37 $\pm$ 3   \\ \hline
   7     & 1270        & 69              & Bank SA             & 0.99857 $\pm$ 0.00003 & 0.99902 $\pm$ 0.00001 & -45 $\pm$ 3   \\ \hline
   8     & 1270        & 134             & Bank SB             & 0.99897 $\pm$ 0.00004 & 0.99932 $\pm$ 0.00001 & -35 $\pm$ 4   \\ \hline
   9     & 1270        & 71              & Bank SC             & 0.99910 $\pm$ 0.00003 & 0.99898 $\pm$ 0.00001 & 12  $\pm$ 3   \\ \hline
   10    & 1270        & 71              & Bank SD             & 0.99916 $\pm$ 0.00003 & 0.99898 $\pm$ 0.00001 & 18  $\pm$ 3   \\ \hline
  \end{tabular}
  \end{adjustbox}
\end{table}

\subsection{Theory}
\subsection{Thermal Expansion Coefficients}
\subsection{On-the-fly Thermal Expansion}
\subsection{Calculation Flow}


%%%%%%%%%%%%%%%%%%%%%%%%%%%%%%%%%%%%%%%%%%%%%
%			section 4
%%%%%%%%%%%%%%%%%%%%%%%%%%%%%%%%%%%%%%%%%%%%%
\newpage 
\section{Numerical Results} 
\subsection{Spatially Continuous Material Properties}
\subsubsection{Two-dimensional Pin-cell Problem}
\subsubsection{Three-dimensional Pin-cell Problem}
\subsubsection{Three-dimensional Assembly Problem}
\subsubsection{Whole-core Reactor Problem}

\subsection{Thermal Expansion}
\subsubsection{Assembly Problem}
\subsubsection{Whole-core Reactor Problem}
\subsubsection{Isothermal Temperature Coefficient}
%\subsubsection{Whole-core Reactor Depletion using Restart Calculations}

\subsection{Combined Framework}
\subsubsection{Assembly Problem}
\subsubsection{Whole-core Reactor Problem}
\subsubsection{Whole-core Reactor Depletion using Restart Calculations}


%%%%%%%%%%%%%%%%%%%%%%%%%%%%%%%%%%%%%%%%%%%%%
%			Conclusion
%%%%%%%%%%%%%%%%%%%%%%%%%%%%%%%%%%%%%%%%%%%%%
\newpage 
\section{Conclusion} 
My conclusion here.

\clearpage

%%%%%%%%%%%%%%%%%%%%%%%%%%%%%%%%%%%%%%%%%%%%%
%			Reference
%%%%%%%%%%%%%%%%%%%%%%%%%%%%%%%%%%%%%%%%%%%%%
\addcontentsline{toc}{section}{References}
\bibliographystyle{IEEEtran}
\bibliography{main.bib}
\clearpage

%%%%%%%%%%%%%%%%%%%%%%%%%%%%%%%%%%%%%%%%%%%%%
%			Acknowledgements
%%%%%%%%%%%%%%%%%%%%%%%%%%%%%%%%%%%%%%%%%%%%%
\addcontentsline{toc}{section}{Acknowledgements}
\section*{\hfill \Large Acknowledgements \hfill}
First and foremost, I am deeply grateful to Allah for His endless mercy and blessings throughout this journey. His grace has given me strength through all the challenges.

I would like to express my sincere thanks to my advisor, Prof. Deokjung Lee, for his invaluable advice, continuous encouragement, and financial support during my study in the CORE Laboratory. I am deeply grateful for the opportunity he provided me to study in his laboratory. Studying and working in the CORE Laboratory has opened new horizons of knowledge for me.

I also want to extend my appreciation to the members of my dissertation committee. Their thoughtful feedback and constructive criticism have greatly helped in refining my work and pushing me to achieve better results. My gratitude also extends to the staff at the Department of Nuclear Engineering at Ulsan National Institute of Science and Technology (UNIST).

I am also grateful to my lab mates for their friendship, insightful discussions, and moral support. Their companionship has been a source of strength during tough times and has made the long hours in the lab much more enjoyable. 

To my beloved parents, I cannot thank you enough. Your unconditional love, support, and prayers have been my constant source of motivation. Your sacrifices and unwavering belief in me are beyond words, and I dedicate this achievement to you.

Finally, I want to thank my family, especially my dear wife. Her love, patience, and strength have been my greatest support throughout this journey. She took care of our four sons while living abroad with me during my studies, and her dedication and sacrifice have made this accomplishment possible.

I would also like to express my thanks to everyone who has supported me in any way during this journey. Your help and encouragement have meant the world to me.
\clearpage

%%%%%%%%%%%%%%%%%%%%%%%%%%%%%%%%%%%%%%%%%%%%%%%%%%%%%%%%%%%%% 
%%%%%%%%%%    Do not touch the below 
%%%%%%%%%%%%%%%%%%%%%%%%%%%%%%%%%%%%%%%%%%%%%%%%%%%%%%%%%%%%% 

%%% The following page is intentionally left as blank
% White attachment form
\hbox{ }
\thispagestyle{empty}
\clearpage

\end{document}

