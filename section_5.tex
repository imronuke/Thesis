\section{Conclusion}  \label{s5}

\subsection{Thesis Summary}

The primary objectives of this thesis were to investigate and develop a framework that enhance the fidelity of current Monte Carlo (MC) multi-physics coupling in reactor simulations by incorporating:
\begin{enumerate}
    \item MC multi-physics coupling with spatially continuous material properties.
    \item On-the-fly thermal expansion (TE) capabilities.
\end{enumerate}

The introduction of MC multi-physics coupling with spatially continuous material properties was facilitated by the use of Functional Expansion Tallies (FET) and delta-tracking techniques. FET was employed to obtain continuous representations of power, which enables the calculation of nearly continuous material properties distributions such as temperature and density. Then by using spatial interpolation, material properties at any spatial points can be calculated. Meanwhile, delta-tracking was utilized as a particle tracking method in a continuous medium, overcoming the limitations posed by traditional surface tracking methods. These enhancements significantly reduce the need for spatial discretization commonly required in MC multi-physics reactor simulations, thereby preserving the advantages of the MC method in handling continuous geometry.

On-the-fly thermal expansion was integrated to address the challenge that the core temperature profile required for TE is not known a priori and is typically non-uniform across the core. By incorporating on-the-fly thermal expansion, the core geometry can be modified on-the-fly during particle tracking based on the local temperatures at the pin-level. This method more closely emulates natural phenomena and is expected to yield more accurate solutions.

Each of these approaches was tested individually on several reactor problems, ranging from pin to whole-core levels, to assess their applicability and advantages. The continuous representation of the intra-fuel pellet profile enabled by MC multi-physics coupling with spatially continuous material properties captures more precise resonance absorption at the fuel-pellet periphery. This leads to a more accurate eigenvalue solution from the proposed framework.

Additionally, incorporating spatially continuous material properties results in solutions that asymptotically converge to those obtained from conventional cell-based discretized simulations with infinitesimally small cells as observed in the two-dimensional pin-cell problem. Similar behavior is observed in three-dimensional pin cell and assembly problems, where the continuous representation of material properties yields more accurate solutions for both eigenvalue and axial power distributions, compared against the conventional cell-based discretization approach. In the whole-core reactor problem, the proposed approach reproduces high-fidelity solutions for both eigenvalue and pin powers, while accelerating the simulation time to almost three times faster and requiring 80\% less memory than the cell-based discretization approach using very small cells.

The numerical test problems with on-the-fly thermal expansion demonstrate that the trends in reactivity differences due to thermal expansion for varying boron concentrations and fuel temperatures, as well as the improvement in calculated ITC with thermal expansion, are consistent with previous studies. While using core-averaged temperatures for expansion is fairly accurate, employing local temperatures can enhance accuracy. Incorporating thermal expansion also improves the solutions for depletion problems, particularly at high power and high fuel burnup. This improvement is evident in the depletion problem using restart case, where modeling thermal expansion significantly improves the critical boron concentration (CBC) at the end of cycle (EOC).

In conclusion, this thesis has successfully demonstrated that the integration of MC multi-physics coupling with spatially continuous material properties and on-the-fly thermal expansion capabilities significantly enhances the accuracy and efficiency of reactor simulations.

\subsection{Future Work}

While the results of the proposed framework are promising, several enhancements are necessary to make the framework suitable for real reactor problems and to further optimize the proposed methods. Here are several recommended areas of development:
\begin{enumerate}
    \item The most natural extension of this framework is the integration of continuous material depletion. Incorporating this feature along with fuel assembly shuffling capabilities would make the framework immediately applicable to real reactor problems. However, accurately representing the continuous radial profile of U-238 absorption rates remains challenging. The U-238 absorption rate is notably high at the periphery of the fuel pellet and decreases steeply towards the center due to spatial self-shielding. Unfortunately, such radial profiles cannot be adequately represented using FET with Zernike polynomials, as FETs are only suited for approximating smooth functions. Using piecewise function, as in the axial direction, is also not a solution because Zernike function is valid for a unit disk, not an annulus. Therefore, alternative polynomial functions, or perhaps completely different approach is necessary to obtain a continuous radial profile for U-238 absorption rates.

    \item Using interpolation to determine material properties at any point  might not be the best approach, particularly for fuel pellets where numerous interpolation points are required. The interpolation points for fuel pellet temperature and xenon distribution are two-dimensional in the radial and axial directions; thus they require more interpolation points, and in turn more memory. The memory requirement is compounded if continuous depletion is implemented, as the interpolation points for hundreds of nuclides' densities in each fuel pellet must also be saved during particle tracking. A potential solution could be to represent continuous fuel pellet temperature and nuclide densities using FET. However, this approach might also necessitate storing numerous FET coefficients, especially in the presences of spacer grids, where the material properties distributions must be represented using piecewise functions. Additionally, as previously mentioned, evaluating material properties at any point using FET requires more arithmetical operations than spatial interpolation. Therefore, the trade-offs associated with using FET to represent continuous material properties must be carefully studied.
\end{enumerate}

These extensions are worth exploring to enhance the practical utility of this framework, enabling it to solve real reactor problems with greater accuracy and improved fidelity through more reasonable simulations. This progress will be instrumental in optimizing reactor design and operation, ensuring safer and more efficient nuclear energy production.