\section{Introduction}

Over the past five decades, advancements in modern computing have enabled direct whole-core Monte Carlo (MC) simulations coupled with multi-physics models for large-scale Light Water Reactors (LWRs). Numerous examples of these reactor simulations can be found in the literature \cite{tung_2020,kelly_2017,ma_2019}. However, according to Smith and Forget \cite{smith_2013}, there are still many important aspects of LWR simulations that must be incorporated to produce truly high-fidelity analysis tools.

While the work of Smith and Forget highlights many aspects to improve the fidelity of reactor calculations, this paper focuses only on two aspects. They are: first, inadequate spatial resolution, such as the improper modeling of radial temperature variations in fuel pellets, which is essential for accurately capturing spatial self-shielding effects \cite{nchoi_2020}. Second, thermal expansion of the reactor core materials, which is often neglected in typical direct whole-core Monte Carlo coupled multi-physics simulations.

The following subsections will discuss each of these aspects in detail.

\subsection{Spatial Resolution}

Multi-physics reactor simulations require discretizing the problem domain into smaller cells, with material composition assumed to be uniform within each cell. This discretization is essential for achieving adequate spatial resolution and accurately modeling variations in material properties, such as fuel temperature, moderator density, and the isotopic composition of materials diluted in the moderator.

Fig. \ref{fig_1} shows the axial view of a typical MC geometry model for a single fuel pin. The actual geometry consists of the fuel, gap, cladding, and coolant, as illustrated in Fig. \ref{fig_1a}. For multi-physics simulations, this geometry must be discretized into several cells both radially and axially, as shown in Fig. \ref{fig_1b}, even though the actual geometry is relatively simple.

However, those additional discretizations can hinder MC particle tracking. This occurs because the cross-sections must be reconstructed each time a particle crosses a cell boundary. Additionally, the routine that determines a particle's location within a cell slows down due to the increased number of cells and must be called more frequently. Moreover, this introduces a significant memory burden, as the data for numerous cells must be stored during the simulation. Consequently, this additional discretization in multi-physics simulations reduces the efficiency of the MC method in handling continuous geometry.

\begin{figure}[h]
    \centering
    \begin{subfigure}[b]{0.25\textwidth}
        \centering
        \includegraphics[width=\textwidth]{figs/sec_1a.png}
        \caption{Actual geometry}
        \label{fig_1a}
    \end{subfigure}
    \hspace{6em}
    \begin{subfigure}[b]{0.25\textwidth}
        \centering
        \includegraphics[width=\textwidth]{figs/sec_1b.png}
        \caption{Discretized geometry}
        \label{fig_1b}
    \end{subfigure}
    \caption{Axial view of a typical Monte Carlo geometry model.}
       \label{fig_1}
\end{figure}

Few methods have been developed to address this issue. One of such method is Localized Delta Tracking (LTD) \cite{nchoi_2020}, which solves radial heat conduction within the fuel pellet using polynomial fitting. This approach allows for the determination of continuously varying radial fuel temperatures. By using LTD within the fuel pellet, MC multi-physics coupling can be achieved without requiring explicit radial discretization of the fuel pellet. However, LTD is limited to handling continuously varying fuel temperatures only in the radial direction. In the axial direction, traditional discretization of the problem domain is still necessary.

One notable effort was made by Ellis \cite{ellis}, who employed the Functional Expansion Tally (FET) method \cite{chadsey,gries} to achieve continuous representations of power. After multi-physics feedback calculations, the resulting continuous fuel pellet temperature and coolant density were also modeled using functional expansions. To facilitate MC particle tracking in a continuous medium, a modified version of Continuously Varying Material Tracking (CVMT) \cite{brown} was developed. These two methods were integrated to perform MC multi-physics simulations with continuously varying materials. However, using CVMT adds extra computational time due to the neutron path integration required during particle tracking simulations. Moreover, the application of this work was limited only up to assembly level.

\subsection{Thermal Expansion}

Thermal expansion (TE) significantly impacts reactor physics modeling, not only for fast reactors but also, to some extent, for LWRs. That is because the reactor design information is typically provided at room temperature, while the actual reactor core operates at much higher temperatures. For instance, at full power, the nominal fuel temperature is around 900 $^{\circ}$C, and the coolant temperature is approximately 300 $^{\circ}$C. This substantial temperature difference causes thermal expansion of all components within the reactor vessel. For example, the core plate expands radially, which increases the assembly pitch. The grid spacers within an assembly also expand, altering the pin pitch within the lattice. Additionally, both the fuel pellet and the fuel rod cladding expand in radial and axial directions \cite{palmtag}.

Thermal expansion modeling has long been integrated into the industry's best-practice two-step methodology for LWR analysis. In this approach, thermal expansion is accounted for during lattice physics calculations to generate the few-group cross sections for specific operating reactor temperatures. However, direct MC coupled multi-physics simulations present a challenge: local temperature variations are determined from thermal-hydraulic feedback, and they are not known a priori. Additionally, these temperatures are typically non-uniform within the reactor core.

To address this challenge, typical TE modeling in direct coupled multi-physics simulations is often achieved by manually adjusting input files to uniformly expand the reactor core geometry and modify material densities. This method typically uses core-averaged nominal temperatures, as demonstrated by Palmtag et al. \cite{palmtag}. In Palmtag's work, thermal expansion was implemented by processing XML input files to uniformly adjust core dimensions and material densities. The input preprocessor, however, only allows thermal expansion at a pre-set, uniform temperature, usually the core-averaged temperature.

Other studies \cite{fiorina,ma_2021,guo} have implemented thermal expansion using thermo-mechanical solvers in Computational Fluid Dynamics (CFD) codes such as OpenFOAM or ANSYS. Generally, these works employed a neutronic solver to obtain power, followed by a CFD code as the thermal-hydraulic (TH) solver and a mechanical solver to perform material deformation due to temperature changes. However, due to their reliance on direct CFD simulations, these approaches are likely suitable only for small cores, like those in heat pipe or modular reactors. The computational requirements for this method would be exponentially more expensive for larger cores.

\subsection{Scope and Objectives}

The objective of this thesis is to present a framework that enhances the fidelity of typical MC coupled multi-physics simulations for LWRs, with focus on Pressurized Water Reactors (PWRs), through two key improvements. First, the introduction of multi-physics simulations with spatially continuous material properties \cite{imron_2024} to address the spatial resolution issue discussed earlier. Second, the incorporation of thermal expansion of reactor core components during MC particle tracking.

MC coupled multi-physics simulations with spatially continuous material properties are achieved using FET in combination with delta-tracking. By using delta-tracking instead of CVMT, the complexity of neutron path integration during particle tracking is eliminated. The proposed method is non-intrusive to the thermal-hydraulic (TH) solver, meaning no modifications are required to the TH solver module. This work also adopts an efficient approach from \cite{honarvar} for constructing Zernike polynomials, marking the first application of this method in the application of FET.

While he incorporation of thermal expansion is achieved through on-the-fly thermal expansion in MC coupled multi-physics reactor simulations. This method dynamically expands the reactor geometry during particle tracking, enhancing accuracy by using local temperatures, such as pin-averaged or assembly-averaged temperatures, rather than core averaged temperatures. Additionally, as will be shown later in this thesis, this approach introduces negligible computational overhead.

The method has been implemented into the MCS code \cite{hlee_2020} to to evaluate the proposed multi-physics framework across a range of reactor core problems, from two-dimensional pin-cell to more realistic whole-core reactor problems. MCS is a neutron/photon transport code that was developed at Ulsan National Institute of Science and Technology (UNIST) with capabilities of performing multi-physics and multi-cycle reactor analyses \cite{hlee_2017,yu_2019,yu_2020}.