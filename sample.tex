%%%%%%%%%%%%%%%%%%%%%%%%%%%%%%%%%%%%%%%%%%%%%%%%%%%%%%%%%%%%%%%%%%
%%%%%%%%%%    Do not touch the below 70 lines     %%%%%%%%%%%%%%%%
%%%%%%%%%%%%%%%%%%%%%%%%%%%%%%%%%%%%%%%%%%%%%%%%%%%%%%%%%%%%%%%%%%
\documentclass[11pt,a4paper,onecolumn,oneside]{report}

\usepackage{mathptmx}
\usepackage[T1]{fontenc}
\usepackage[utf8]{inputenc}

\RequirePackage[top=3cm, bottom=1in, left=1in, right=1in]{geometry}
\linespread{1.3}

\usepackage{titlesec}
\usepackage{amsmath}
\usepackage{amssymb}
\usepackage{mathtools}
\usepackage{enumerate}
\usepackage{bbm}
\usepackage{algorithm}
\usepackage{algorithmic}
\usepackage{epsfig}
\usepackage{color}
\usepackage{graphicx}
\usepackage{caption}
\usepackage{subcaption}
\usepackage{cases}
\usepackage{url}
\usepackage{cite}
\usepackage{fancyhdr}
\usepackage{tocloft}
\usepackage{pdfpages}
\usepackage{adjustbox}
\usepackage{tocloft}

\newlength{\mylen}

\renewcommand{\cftfigpresnum}{\figurename\enspace}
\renewcommand{\cfttabpresnum}{\tablename\enspace}
\renewcommand{\cftfigaftersnum}{:}
\renewcommand{\cfttabaftersnum}{:}
\settowidth{\mylen}{\cftfigpresnum\cftfigaftersnum}
\settowidth{\mylen}{\cfttabpresnum\cfttabaftersnum}
\addtolength{\cftfignumwidth}{\mylen}
\addtolength{\cfttabnumwidth}{\mylen}

\renewcommand\cftsecafterpnum{\vskip15pt}
\renewcommand\cftsubsecafterpnum{\vskip15pt}
\renewcommand\cftfigafterpnum{\vskip15pt}
\renewcommand{\thesection}{\Roman{section}.}
\renewcommand{\thesubsection}{\arabic{section}.\arabic{subsection}.}
\renewcommand{\contentsname}{\hfill\bfseries\Large Contents\hfill}   
\renewcommand{\listfigurename}{\hfill\bfseries\Large List of Figures\hfill}
\renewcommand{\listtablename}{\hfill\bfseries\Large List of Tables\hfill}
\renewcommand{\thefigure}{\arabic{figure}}
\newcommand{\qed}{\hfill\blacksquare}
\renewcommand{\bibname}{\hfill\bfseries\Large References \hfill\hfill}
\renewcommand{\abstractname}{\bfseries\Large Abstract \hfill\hfill}

\captionsetup{figurename=Figure,labelsep=period}
\captionsetup{tablename=Table,labelsep=period}

\newcounter{lemma}
\newcounter{proposition}
\newcounter{theorem}
\newtheorem{lemma}{\bf Lemma}
\newtheorem{proposition}{\bf Proposition}
\newtheorem{theorem}{\bf Theorem}
\newtheorem{proof}{\bf Proof}

%\input{mymath_mod.tex}


\newcommand{\HIGH}[1]{{\textcolor{blue}{#1}}}
%\renewcommand{\baselinestretch}{1.5} 

\DeclareMathOperator*{\argmax}{arg\,max}

\fancyhf{}
\renewcommand{\headrulewidth}{0pt}
\cfoot{\thepage}
\pagestyle{fancy}
%\pagenumbering{gobble}

\begin{document}

%%%%%%%%%%%%%%%%%%%%%%%%%%%%%%%%%%%%%%%%%%%%%%%%%%%%%%%%%%%%%%%%%%
%%%%%%%%%%    Do not touch the above 70 lines     %%%%%%%%%%%%%%%%
%%%%%%%%%%%%%%%%%%%%%%%%%%%%%%%%%%%%%%%%%%%%%%%%%%%%%%%%%%%%%%%%%%


%%%%%%%%%%%%%%%%%%%%%%%%
%%%%%% Front cover
%%%%%%%%%%%%%%%%%%%%%%%%
\begin{center}
\LARGE Master's Thesis / Doctoral Thesis
% or Ph.D...

\vspace{3cm}
\huge <Title>
% Great Work That None Can Do Except Me

\vfill

\LARGE <Your name>
% Gil Dong Hong

\vspace{2cm}

\LARGE <Your department> 
% Department of Computer Science and Engineering

\LARGE (<Your major>)
% (Check your major. Write only if your major name is different from your department(school))

\vspace{2cm}

\LARGE Ulsan National Institute of Science and Technology
\vspace{2cm}

\LARGE <Year>
% 2021

\end{center}
\thispagestyle{empty}
\clearpage

%%%%%%%%%%%%%%%%%%%%%%%%
%%%%%% title page
%%%%%%%%%%%%%%%%%%%%%%%%
\begin{center}
\hbox{ }

\hbox{ }

\huge <Title> 
% Great Work That None Can Do Except Me

\vspace{5cm}

\LARGE <Your name>
% Gil Dong Hong

\vspace{6cm}

\LARGE <Your department> 
% Department of Computer Science and Engineering

\LARGE (<Your major>)
% Check your major. Write only if your major name is different from your department(school)

\vspace{2cm}

\LARGE Ulsan National Institute of Science and Technology

\end{center}
\thispagestyle{empty}
\clearpage

% [Thesis approval]
% Add the approval doc signed by your advisor in a PDF file
% Put your pdf with the filename below, and uncomment it.
%\includepdf[fitpaper= true, pages=-]{sample_approval.pdf}

% [Confirmation of thesis approval]
% add the certificate signed by your committee in a PDF file
% Put your pdf with the filename below, and uncomment it.
%\includepdf[fitpaper= true, pages=-]{sample_confirmation.pdf}

% Abstract
\begin{abstract}
Your abstract should be here. \vfill
\end{abstract}

\clearpage

%%%%%%%%%%%%%%%%%%%%%%%%%%%%%%%%%%%%%%%%%%%%%%%%%%%%%%%%%%%%% 
%%%%%%%%%%    Do not touch the below 20 lines 
%%%%%%%%%%%%%%%%%%%%%%%%%%%%%%%%%%%%%%%%%%%%%%%%%%%%%%%%%%%%% 
%%% The following page is intentionally left as blank
%%% White attachment form
\hbox{ }
\thispagestyle{empty}
\clearpage

%%% Table of Contents
\tableofcontents{}
\thispagestyle{empty}
\vfill
\clearpage

%%% List of Figures
\listoffigures
\thispagestyle{empty}
\clearpage

%%% List of Tables
\listoftables
\thispagestyle{empty}
\clearpage

%%% reset page numbering
\setcounter{page}{1}
%%%%%%%%%%%%%%%%%%%%%%%%%%%%%%%%%%%%%%%%%%%%%%%%%%%%%%%%%%%%% 
%%%%%%%%%%    Do not touch the above 20 lines 
%%%%%%%%%%%%%%%%%%%%%%%%%%%%%%%%%%%%%%%%%%%%%%%%%%%%%%%%%%%%% 



%%%%%%%%%%%%%%%%%%%%%%%%%%%%%%%%%%%%%%%%%%%%%
%			section 1, usually Introduction
%%%%%%%%%%%%%%%%%%%%%%%%%%%%%%%%%%%%%%%%%%%%%
\section{Introduction} 

Your introduction should be here. You may need a reference~\cite{ref_sample}.


%%%%%%%%%%%%%%%%%%%%%%%%%%%%%%%%%%%%%%%%%%%%%
%			section 2
%%%%%%%%%%%%%%%%%%%%%%%%%%%%%%%%%%%%%%%%%%%%%
\newpage 
\section{Second section} 

My second section starts with my equation, which can be written as 
%
\begin{equation}\label{eq:myeq}
\begin{split}
	A 		&= B + C, \\
    D + E	&= F.
\end{split}
\end{equation}
Thus we have (\ref{eq:myeq}).

Fig.~\ref{fig:myfigure} is a sample figure. 

\begin{figure}[h]
\centering
\includegraphics[width=5in]{myfigure.pdf}
\caption{My figure.} \label{fig:myfigure}
\end{figure}

%%%% use the following in the case of eps figure
%\begin{figure} 
%    \centering
%    \epsfig{file=model.eps, width = 0.9\linewidth}
%    \caption{System model with complete bipartite graph. The maximum weighted matching is marked by circles.}
%    \label{fig:model}
%\end{figure}



%%%%%%%%%%%%%%%%%%%%%%%%%%%%%%%%%%%%%%%%%%%%%
%			section 3
%%%%%%%%%%%%%%%%%%%%%%%%%%%%%%%%%%%%%%%%%%%%%
\newpage 
\section{Third section} 
My third section starts. 

\subsection{First subsection}
This is the first subsection. 


\begin{algorithm}
	\caption{My Algorithm.} \label{algo:myalgo}
    At the beginning ...
    
	\begin{algorithmic}[1]	    
	    \STATE Do this
        \STATE and do this\\
        /* add explanation if necessary */
        \STATE Finally do this
	\end{algorithmic} 
\end{algorithm}
%
Algorithm~\ref{algo:myalgo} is our proposed algorithm.

\begin{table}[!b]
  \centering
  \caption{Solutions for the criticality calculations}
  \label{tab1} 
\begin{adjustbox}{width=1.0\textwidth} % Adjust your table to the text width
  \begin{tabular}{| p{0.05\linewidth} | p{0.1\linewidth} | p{0.1\linewidth} | p{0.1\linewidth} | c | c | p{0.1\linewidth} |}
  \hline 
         &             &                 &                     & \multicolumn{2}{c|}{$k_{eff}$}             &            \\
   \cline{5-6}
   Cases & Boron (ppm) & Bank D Position (steps) & Fully inserted bank & MCS                   & KENO                  & Difference (pcm) \\
   \hline
   1     & 1285        & 167             & -                   & 0.99954 $\pm$ 0.00002 & 0.99990 $\pm$ 0.00001 & -36 $\pm$ 2   \\ \hline
   2     & 1291        & 230             & -                   & 0.99998 $\pm$ 0.00003 & 1.00032 $\pm$ 0.00001 & -34 $\pm$ 3   \\ \hline
   3     & 1270        & 97              & Bank A              & 0.99836 $\pm$ 0.00003 & 0.99880 $\pm$ 0.00001 & -44 $\pm$ 3   \\ \hline
   4     & 1270        & 113             & Bank B              & 0.99900 $\pm$ 0.00003 & 0.99936 $\pm$ 0.00001 & -36 $\pm$ 3   \\ \hline
   5     & 1270        & 119             & Bank C              & 0.99863 $\pm$ 0.00003 & 0.99904 $\pm$ 0.00001 & -41 $\pm$ 3   \\ \hline
   6     & 1270        & 18              & Bank D              & 0.99871 $\pm$ 0.00003 & 0.99908 $\pm$ 0.00001 & -37 $\pm$ 3   \\ \hline
   7     & 1270        & 69              & Bank SA             & 0.99857 $\pm$ 0.00003 & 0.99902 $\pm$ 0.00001 & -45 $\pm$ 3   \\ \hline
   8     & 1270        & 134             & Bank SB             & 0.99897 $\pm$ 0.00004 & 0.99932 $\pm$ 0.00001 & -35 $\pm$ 4   \\ \hline
   9     & 1270        & 71              & Bank SC             & 0.99910 $\pm$ 0.00003 & 0.99898 $\pm$ 0.00001 & 12  $\pm$ 3   \\ \hline
   10    & 1270        & 71              & Bank SD             & 0.99916 $\pm$ 0.00003 & 0.99898 $\pm$ 0.00001 & 18  $\pm$ 3   \\ \hline
  \end{tabular}
  \end{adjustbox}
\end{table}

\subsection{Another subsection}


%%%%%%%%%%%%%%%%%%%%%%%%%%%%%%%%%%%%%%%%%%%%%
%			sections...
%%%%%%%%%%%%%%%%%%%%%%%%%%%%%%%%%%%%%%%%%%%%%
\newpage 
\section{Following Sections} 


%%%%%%%%%%%%%%%%%%%%%%%%%%%%%%%%%%%%%%%%%%%%%
%			Conclusion
%%%%%%%%%%%%%%%%%%%%%%%%%%%%%%%%%%%%%%%%%%%%%
\newpage 
\section{Conclusion} 
My conclusion here.

\clearpage

%%%%%%%%%%%%%%%%%%%%%%%%%%%%%%%%%%%%%%%%%%%%%
%			Reference
%%%%%%%%%%%%%%%%%%%%%%%%%%%%%%%%%%%%%%%%%%%%%
\addcontentsline{toc}{section}{References}
\bibliographystyle{IEEEtran}
\bibliography{sample.bib}
\clearpage

%%%%%%%%%%%%%%%%%%%%%%%%%%%%%%%%%%%%%%%%%%%%%
%			Acknowledgements
%%%%%%%%%%%%%%%%%%%%%%%%%%%%%%%%%%%%%%%%%%%%%
\addcontentsline{toc}{section}{Acknowledgements}
\section*{\hfill \Large Acknowledgements \hfill}
Thank you very much.
\clearpage

%%%%%%%%%%%%%%%%%%%%%%%%%%%%%%%%%%%%%%%%%%%%%%%%%%%%%%%%%%%%% 
%%%%%%%%%%    Do not touch the below 
%%%%%%%%%%%%%%%%%%%%%%%%%%%%%%%%%%%%%%%%%%%%%%%%%%%%%%%%%%%%% 

%%% The following page is intentionally left as blank
% White attachment form
\hbox{ }
\thispagestyle{empty}
\clearpage

\end{document}

